\documentclass{article}
	\title{WorldEditArt}
	\date{2017-07-20}
	\author{SOFe}

\iffalse
	\usepackage{helvet}
\fi
\renewcommand{\familydefault}{\sfdefault}

\def \code #1{\texttt{#1}}

\begin{document}
	\maketitle
	This proposal describes a world edit plugin largely based on features originally intended in WorldEditArt, along with
	some new ideas.
	\section{Plugin mechanism}
		\subsection{Builder sessions}
			Users of this plugin should be managed in "builder sessions" with individual access information. Builder
			sessions can be created in three modes, namely:
			\paragraph{Implicit Mode}
				Players with a certain permission will start a builder session upon joining a game. The session's permission
				and location will be synchronized with the player.
			\paragraph{Explicit Mode}
				Players with a certain permission will start a builder session upon typing a command. The session's
				permission and location will be synchronized with the player, and can be closed with a command. The command
				may be locked with a private password (similar to the \texttt{sudo} Linux command) or global password
				(similar to the \texttt{su} Linux command) for additional protection.
			\paragraph{Minion Mode}
				Command senders with a certain permission (especially non-in-game senders like console) can create minion
				builder sessions upon typing a command. The session's permission and location will be controlled by the
				command sender.
			Each builder session has an allocated amount of resources; this allocation may affect the rate of world-editing
			operations to maximize server performance.

			For implicit and explicit modes, the builder session's position and orientation is synchronized with the
			player. The position uses the block that \textbf{the player's feet stand in}.
			\begin{itemize}
				\item If the player is floating on a lake of liquid, the highest level of the liquid is used.
				\item If the player is standing on a full block (or the part of the block with full height, e.g. an upper slab, the upper step of a stair block), the \emph{air block above} the full block is used.
				\item If the player is standing on an incomplete block (e.g. a lower slab, a chest block, an open-upwards trapdoor), the block itself is used.
			\end{itemize}

		\subsection{The core world editing chain}
			\begin{lstlisting}
				Selection -> Block Iterator -> Block Changer -> Cassette -> User History -> Load Synchronizer
			\end{lstlisting}

\end{document}
